\chapter*{Závěr}\label{chap:zaver}\addcontentsline{toc}{chapter}{Závěr}

Hlavním souhrnným cílem této bakalářské práce byl \textbf{navrhnout systém pro dálkové ovládání dopravníků} pro společnost Honeywell – specificky pro použití při dynamických kontrolách instalace dopravníků. Primárním cílem bylo vytvořit jednoduché a bezpečné řešení, které je možné ovládat i z bezprostřední blízkosti, ale i prostřednictvím mobilního zařízení pomocí WiFi komunikace.

V úvodní části práce byly položeny teoretické základy nezbytné pro pochopení návrhu systému. Nejprve bylo prozkoumáno, jakým způsobem se řídí rychlost dopravníků a jak frekvenční měniče fungují. Pozornost byla také věnována analýze specifického frekvenčního měniče \textbf{Sinamics G120D}, pro který byl tento návrh primárně určen. Důležitou součástí této analýzy byly možnosti nastavení ovládacího panelu, díky kterým celý systém komunikuje s frekvenčním měničem. V rámci rešerše byla také prozkoumána možnost integrace mikrokontroléru do návrhu, k čemuž byla zvolena vývojová deska \textbf{WEMOS D1 Mini Pro} s mikrokontrolerem \textbf{ESP8266EX}. Tato deska poskytuje dostatečné parametry s integrovanou WiFi komunikací a možností připojit externí WiFi anténu. Společně s touto deskou bylo prozkoumáno také, jakým způsobem je možné desku programovat prostřednictvím Arduino frameworku pro ESP8266.

Druhá kapitola se věnovala návrhu celého systému – nejprve z hlediska architektury a požadavků, a následně návrhu jednotlivých dílčích částí. Tři rovnocenné části systému jsou:
\begin{itemize}
	\item \textbf{Uživatelská mobilní aplikace}, pomocí které je jednoduše možné ovládat a nastavit dopravník.
	\item \textbf{Firmware vývojové desky}, který pomocí ESP8266 WebServeru interpretuje instrukce z mobilní aplikace.
	\item \textbf{Elektronika}, která s navrženou deskou plošných spojů převádí výstupy z vývojové desky na instrukce pro ovládací panel frekvenčního měniče.
\end{itemize}
V rámci návrhové kapitoly byl popsán proces tvorby obvodu pro komunikaci s ovládacím panelem frekvenčního měniče, včetně některých zajímavých problémů, které při tvorbě nastaly. Následně byl hardware integrován do desky plošných spojů. V části zabývající se firmwarem vývojové desky byl popsán objektově orientovaný přístup při programování a jeho výhody. Následně se tato část zaměřila na stavový diagram logiky systému a aproximaci rychlosti dopravníku. Po popisu firmwaru se tato kapitola věnovala sekci týkající se vývoje mobilní aplikace. Tam byla nejprve popsána architektura aplikace a následně byla do hloubky rozebrána komunikace aplikace a WebServeru. Pozornost byla také věnována designu mobilní aplikace. Na konci návrhové části bylo věnováno místo návrhu schránky pro desku plošných spojů a kompletaci hardwaru.

Závěrečná kapitola se zabývala ověřením navrženého systému. To bylo provedeno v brněnské hale společnosti Honeywell – bylo tedy možné systém otestovat v prostředí, které je velmi podobné halám zákazníků, kde bude systém skutečně používán. Při testování bylo potvrzeno, že je systém možné ovládat jak z lokálního ovládání umístěného na schránce, tak i v režimu dálkového řízení pomocí mobilní aplikace. Testována byla také spolehlivost bezdrátové komunikace, přičemž byl naměřen uspokojivý dosah 35 metrů a bylo potvrzeno, že je možné ovládat více jednotek pomocí jedné mobilní aplikace. Zhodnocena byla také bezpečnost navrženého systému, kde bylo vyhodnoceno, že systém spoléhá na stávající, již nainstalované bezpečnostní prvky dopravníkových linek. Testy potvrdily správnou implementaci a funkčnost navrženého systému v podmínkách, které se blíží reálnému provozu.

Závěrem lze říci, že byly splněny všechny cíle stanovené pro tuto bakalářskou práci, jelikož se týkaly návrhu tří hlavních částí tvořících systém (Hardware, Firmware a Software) a následného otestování funkčnosti navrženého systému. Díky této bakalářské práci tedy vznikl systém, který může inženýrům společnosti Honeywell usnadnit dynamické testování dopravníků.